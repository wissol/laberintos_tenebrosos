\documentclass[10pt,spanish,DIV=10,twoside=false]{scrartcl}

% ------ tipografía ---------------------
\usepackage[spanish,es-noindentfirst]{babel}
\usepackage[utf8]{inputenc} % --- codificación de caracteres (entrada)
\usepackage[T1]{fontenc}
\usepackage{Alegreya} 			% --- U otra fuente que se desee
\renewcommand*\oldstylenums[1]{{\AlegreyaOsF #1}}

\usepackage[T1]{fontenc}


\addtokomafont{disposition}{\rmfamily} 	% --- Encabezados en Serif
\addtokomafont{descriptionlabel}{\rmfamily}

\usepackage{microtype}  % --- Arreglos Tipográficos
\usepackage{setspace} % --- Doble espacio, usado en haikus
\usepackage{threeparttable}
\setlength{\columnsep}{1cm}
\usepackage[autostyle]{csquotes}
\usepackage{hyperref}
\usepackage{booktabs}
\usepackage{enumitem}
\usepackage{longtable}

%\pagestyle{plain}
\usepackage{graphicx}
\usepackage{fixltx2e}
\usepackage{multicol}


\makeindex


\begin{document}

\title{Laberintos Tenebrosos}
\subtitle{Juego de Rol Básico}
\author{Miguel de Luis}
\dedication{Para Alberto, Alexis, Dimas, Luis y Quintín y quienes la magia amen\dots}
\date{\today}
\maketitle
\begin{multicols}{2}
\tableofcontents
\end{multicols}
\pagebreak
\vfill
\begin{multicols}{2}

\listoftables

\end{multicols}
\vfill
\pagebreak


\part{Reglas}

\begin{abstract}
\noindent \enquote{Laberintos Tenebrosos} es un juego de rol tradicional que se inspira en
las versiones más clásicas y queridas del juego de rol de fantasía más popular
de todos los tiempos, pero simplificando las reglas todo lo posible sin sacrificar nada esencial.

En este juego los personajes solo pueden llegar hasta nivel 6, pero eso es
suficiente para varios meses de aventuras.
\end{abstract}

\vspace{1cm}

\begin{multicols}{2}



\includegraphics[width=\columnwidth]{sceata.png}

\section{La regla principal}

El máster cuenta una aventura a los demás jugadores. Cada uno de estos jugadores
lleva a uno de los protagonistas de la aventura, los personajes-jugadores.
Los personajes-jugadores pueden, ---dentro de lo que permitan las reglas---,
\emph{intentar} cualquier cosa que deseen.

El máster determina cuáles han sido las consecuencias de esas acciones y describe
que pasa a continuación. Y así sucesivamente hasta que todos los personajes mueren
o, por las razones que sean, los jugadores consideran que la aventura está terminada.

\begin{quotation}
Os adentráis en el bosque, cuando, entre las sombras, descubrís el brillo de una
espada. ¿Qué hacéis?

---¡Atacamos!

---!No!, salgamos de aquí.

---Prendamos fuego al bosque.

---Intento descubrir quiénes son.

---¿Por qué no hablamos con ellos?
\end{quotation}

Esto es todo el juego de rol y el resto, comentarios.

\section{Dados}

Los dados de este juego son $d20$, $d12$, $d10$, $d8$, $d6$ y $d4$. Se llaman así por el
número de caras que tienen. El dado \emph{normal} es el $  d6$, de 6 caras de toda
la vida.

También hablamos de dos dados que no existen en la realidad: el $d2$ y el $d3$. Para simular
un $d2$ lanza un dado normal, los impares son 1, los pares un 2. Para el $d3$ lanza un $d6$,
1-2 es 1, 3-4 es 2 y 5-6 es 3.

\section{Atributos}\label{}

Los personajes de este juego cuentan con los siguientes atributos: Fuerza (FUE),
Destreza (DES), Constitución (CON), Inteligencia (INT), Sabiduría (SAB) y
Carisma (CAR).

Lanza $3d6$ (3 dados de 6 caras) para cada uno de los atributos, por su orden.
Anota esos valores en la hoja de personaje y también el modificador, según el
cuadro de modificadores, en la página \pageref{c_mods_atributos}.

El modificador de atributos se usa para tres cosas: Modificar la tirada de Ataque,
el daño que causan las armas, tu clase de armadura y los puntos de vida, como se
explica más adelante.

\begin{table*}[p]
\centering
\label{c_mods_atributos}
\caption{Modificares de atributos}
\begin{tabular}{lcccccc}
\toprule
Valor       &  3-5  &  6-8 & 10-12 & 13-15 & 16-17 & 18\\
\midrule
Mod. &  -2   &  -1  &  -    &  +1   &  +2   & +3\\
\bottomrule
\end{tabular}
\end{table*}

\section{Clases de Personajes}

La versión básica de \enquote{Laberintos Tenebrosos} cuenta con las siguientes
clases de personaje: Guerrero, Paladín, Mago, Ladrón, Elfo, Enano y Hobbit. La clase de
personaje determina los puntos de vida de tu personaje, qué armas y armadura
puede emplear y si dispone de habilidades especiales.

El guerrero es un experto en el uso de las armas y la vida en el campo, pero
poco más. Un guerrero solo sabe leer si su inteligencia es superior a 9. El Mago
es un experto en la magia y, además, conoce gran variedad de lenguas y todo tipo
de ciencias y conocimientos. El ladrón es una criatura de ética peculiar, muy
interesado en el dinero de los demás y con habilidades especiales para estos
lucrativos fines. El elfo, hobbit y enanos son criaturas de leyendas, escasas en
los reinos humanos, pero con poderes y características especiales.

\minisec{El Paladín}
En cuanto al Paladín, es una criatura aparte. Los sabios creen que los paladines
vienen al mundo re-encarnados de otro que llaman simplemente Tierra o "La Tierra".
Dichos paladines tienen como único fin en la vida hacer el bien y, por ello, hacen
las siguientes promesas la edad de doce años y no se separan de ellos de por vida.

\minisec{Promesas}
\begin{itemize}
\item Caridad. Un paladín debe donar todo el dinero que le sea posible a instituciones
de Caridad como orfanatos, templos, asilos, etc.
\item Verdad. Un paladín nunca puede mentir. Nunca.
\item Misericordia. Un paladín nunca puede hacer daño a una criatura indefensa,
o a un monstruo. Debe intentar preservar la vida de todos, incluso de un goblin.
La única excepción son los demonios, que no pueden convertirse al bien.
\end{itemize}

\minisec{Puntos de Vida}

Añade ahora tus puntos de vida a tu hoja de personaje. Como personaje de primer nivel empiezas con 1
punto de vida por tu nivel. Añade a eso el Modificador que tengas por el
Atributo de Constitución. Por último lanza el dado de puntos de vida que
corresponda a tu clase\footnote{Ver pág \pageref{c:clases}}.

Por ejemplo eres un Mago con Constitución 13. Tienes 1 por tu nivel, +1 por tu
modificador de atributo. Ahora lanzas el 1d4, que es el que corresponde al mago
y sacas un 3. Tu total es 5 ($1 + 1 + 3 = 5$).

Si el resultado es 0, tendrías 1 punto de vida de todas maneras.


\begin{table*}[p]
\begin{threeparttable}
\label{c:clases}
\caption{Clases de Personaje}
\begin{tabular}{lccccp{3.7cm}}
\toprule
Clase & Vida & Armadura & Armas & Magia & Especiales\\
\midrule
\midrule
Guerrero &$d8$ & Todas & Todas & - & - \\
Mago &$  d4$ & Ninguna & Daga, Honda y Cayado & Sí & - \\
Paladín &$  d8$ & Todas & Cuerpo a Cuerpo & Sí & Curar \\
Ladrón &$  d6$ & Gambesón$^*$ & Cortas & - & †\\
Elfo &$  d6$ & Cota de Malla y Gambesón & Cortas y Arco Largo & Sí & ††\\
Enano &$  d6$ & Todas & Cuerpo a cuerpo y Ballesta & - & Infravisión, Detectar Oro \\
Hobbit &$  d4$ & Cota de Malla y Gambesón$^*$ & Cortas & - & Susurros, Silencio \\
\bottomrule
\end{tabular}
\begin{tablenotes}
  \item * No puede usar escudo ni casco
  \item †  Trampas, Cerraduras, Susurros, Trepar Paredes Lisas
  \item †† Infravisión, Trampas, Susurros, +3 a las tiradas relacionadas con la visión
  \item Los cuadros de armas y armaduras especifican qué armas y armaduras concretas puede usar cada clase
\end{tablenotes}
\end{threeparttable}
\end{table*}


\section{Habilidades especiales}

\minisec{Cerraduras} Permite al ladrón abrir cerraduras con ganzúas, u otra
herramienta improvsiada. La mayoría de las cerraduras tienen una dificultad típica
para abrirlas, que se ve en el Cuadro de Dificultades Típicas.

Los personajes que no tengan esta habilidad no pueden intentar abrir cerraduras
de esta manera, sino que deberán buscar una llave o emplear la fuerza.

\minisec{Curar} Permite al paladín Curar $ 1d6$ punto de vida por nivel, a cualquier
personaje o criatura que no sea él mismo. Puede usar esta habilidad una vez al día.

\minisec{Detectar Oro} Permite al enano detectar oro o plata hasta una distancia
de 20 metros. A esta habilidad no le afecta que existan muros u otros obstáculos
y solo puede impedirse si el tesoro está protegido con magia.

\minisec{Infravisión} Permite a los elfos y enanos ver en la oscuridad hasta
20 metros de distancia como si dispusieran de visión infrarroja. Esto es ven las
figuras como fuentes de calor. Esta habilidad no funciona a través de muros u
otros obstáculos sólidos de más de 20 centímetros de grosor.

\minisec{Silencio} Permite al hobbit moverse, incluso corriendo, sin hacer
ningún ruido en absoluto. Ni siquiera puede detectarse con Susurros. Esta habilidad
requiere una tirada de Destreza con tanta dificultad como la del nivel del monstruo
de nivel superior que no queremos que nos detecte.

\minisec{Susurros} Permite al ladrón al elfo y al hobbit escuchar ruidos muy sutiles,
que ninguna otra persona sería capaz. También permite comunicarse con palabras
que solo otro elfo (o ladrón puede escuchar). En ambos casos el alcance es hasta
20 metros, y no funciona a través de muros u otros obstáculos sólidos de más de 20
centímetros de grosor.

\minisec{Trampas} Permite al ladrón descubrir y desactivar trampas. Hace
falta una acción (y una tirada) distinta para cada acción. Descubrir una trampa
es una tirada de Sabiduría, mientras que desactivarlas es de Destreza.

Los personajes que carezcan de esta habilidad pueden intentar desactivar una
trampa que esté visible o que haya sido descubierta por un ladrón. Pero lo hacen
sumando 3 a la dificultad típica y no pueden añadir su nivel.

El fracaso intentando desactivar una trampa hace que se active.

\minisec{Trepar Paredes Lisas} Permite al ladrón trepar por cualquier pared,
por lisa que sea, sin emplear herramientas de escalada. El uso de esta habilidad
requiere una Tirada de Destreza con una dificultad de 1 por cada tres metros de
altura. Un fallo hace que el ladrón desista a mitad de la subida, una pifia
supone una caída.

El uso de herramientas de escalada permite al ladrón reducir la dificultad a la
mitad. El resto de los personajes solo puede subir con herramientas de escalada,
pero con la dificultad que tendrían normalmente.

\begin{table*}[p]
\centering
\caption{Dificultades típicas}
\label{t:dif_tip}
\begin{tabular}{lcccccc}
\toprule
Niveles    & 1 & 2 & 3 & 4 & 5 & 6\\
\midrule
Cerraduras & 2 & 3 & 5 & 6 & 8 & 9\\
Trampas    & 3 & 5 & 7 & 8 & 9 & 10\\
\bottomrule
\end{tabular}
\end{table*}

\includegraphics[width=\columnwidth]{escudo.png}

\section{Compra tu Equipo}

Tira $ 3d6$ x 10 (Tres dados de 6 caras y multiplica el resultado por 10). Ese es
el total de monedas de oro que tu personaje tiene al comenzar la partida. Puedes
gastar todo ese dinero en comprar el equipo necesario. Los precios y otros
detalles vienen en el Cuadro de Equipo.

No te olvides de comprar tus armas y armaduras. Anota todo en tu hoja de personaje
y, especialmente, tu clase de Armadura. Recuerda que tu clase de armadura es igual
a la de armadura que lleves más tu modificador por Destreza.

1 moneda de oro (mo) = 100 monedas de plata (mp) = 1000 monedas de cobre (mc)

\begin{table*}[p]
\centering
\begin{threeparttable}
\caption{Armas}
\begin{tabular}{lccccc}
\toprule
Arma  &  Coste & Daño & Peso & Manos & Clases\\
\midrule
\midrule
Hacha de Mano & 6mo & $ 1d8$ & 1 & 1 & a\\
Ballesta      & 3mo & $ 1d8$ & 2 & 2 & b\\
Daga  & 3mo & $ 1d4$ & 0,5 & 1 & T \\
Arco Largo & 40 mo & $ 1d8$ & 1 & 2 & c \\
Arco Corto & 7mo & $ 1d6$ & 1 & 2 & b \\
Maza &  5mo &  $ 1d6$ &  1 & 1 & a\\
Alabarda & 7mo & $1d10$ & 7 & 2 & d\\
Cayado & 1mp & $ 1d6$ & 2 & 2 & T\\
Honda & 5mc & $ 1d4$ & - & 2 & T\\
Espada Corta & 7mo &  $ 1d6$ & 1 & 1 & a \\
Espada & 10mo & $ 1d8$ & 1 & 1 & e\\
Lanza & 1mo &  $ 1d6$ & 1 & 1 & e\\
Mandoble & 20mo & $ 3d4$ & 3 & 2 & f \\
Sin Armas & - & $ 1d2$ & - & - & T\\
\bottomrule
\end{tabular}

\begin{tablenotes}
  \item T Todos
  \item a Todos menos el mago
  \item b Guerrero, Ladrón, Enano y Hobbit
  \item c Guerrero y Elfo
  \item d Guerrero y Enano
  \item e Guerrero, Paladín, Elfo y Enano
  \item f Guerrero
\end{tablenotes}
\end{threeparttable}
\end{table*}

\begin{table*}[p]
\centering
\caption{Equipo}
\begin{tabular}{lccp{5cm}}
\toprule
Objeto & Coste & Peso & Notas\\
\midrule
\midrule
Mochila&2mo&1 & 12 objetos\\
Manta&1mp&1 \\
Botella de cristal&2mo&-\\
Velas(10)&10mc&-\\
Cadenas(3m)&30mo&1 \\
Palanqueta&2mo&2 \\
Yesca y pedernal&1mp&-&Para hacer fuego\\
Garfio de escalada&1mo&-\\
Martillo&5mp&-& Pequeño\\
Símbolo Sagrado&25mo&1 \\
Tintero & 8mo & \\
Pluma & 1mp & \\
Lámpara de aceite & 9mo & 1 \\
Esposas & 15mo & 1 & Tirada FUE con dificultad 5 \\
Petróleo & 1mp & 0,5 \\
Papiro (1 hoja) & 1mp & -\\
Pico & 3mo & 4\\
Vara 3 metros & 2mp & 2\\
Raciones 1 día & 2mp & 1 & Por persona\\
Cuerda (15 metros) & 1mo & 2,5\\
Pala & 2mo & 4 s\\
Piquetas (12) & 1mo & 4 & Para escalar\\
Ganzúas & 30mo & 0,5\\
Antorchas (8) & 3mp & 4\\
Agua Bendita (1 frasco) & 25mo & -\\
Flechas(12) & 1mp & 1\\
Aljaba & 1mp & 1\\
Saco & 5mc & 1 & Para piedra de honda\\
\bottomrule
\end{tabular}
\end{table*}

\begin{table*}[p]
\centering
\begin{threeparttable}
\caption{Clase de Armaduras}
\begin{tabular}{lcc}

\toprule
Armadura  & Clase & Coste  \\
\midrule
\midrule
Ropas Normales & 9 & 1mp \\
Gambesón & 7 & 30mo \\
Cota de Malla & 5 & 100mo \\
Armadura de Placas & 3 & 200mo \\
Escudo & -2 & 10mo \\
Casco & -1 & 10mo \\
\bottomrule
\end{tabular}

\begin{tablenotes}
\item Los escudos y cascos no tienen una clase de armadura por si mismo sino que
modifican la clase de otra armadura. Por ejemplo si solo llevas ropas normales
pero tienes casco y escudo, --como era normal en los guerreros más pobres de la
Edad Media---, tu clase de armadura es 9 - 2 (escudo) - 1 (casco) = 6.

\end{tablenotes}
\end{threeparttable}
\end{table*}


\section{Acciones}

Los personajes pueden intentar dos acciones por turno, pero solo una de ellas
puede ser un ataque.

Excepto los ataques y los conjuros, --que tienen sus reglas especiales--, todas
las demás acciones se resuelven mediante una prueba del atributo más adecuado,
según decida el máster.

Para hacer una prueba de atributo, el jugador lanza $3d6$, le resta su nivel y le
suma la dificultad. Si el total es menor que el atributo, ha pasado la
prueba, si es mayor ha fracasado.

Por ejemplo: Supongamos que un personaje de nivel 1 quiere derribar una puerta a
patadas. El máster decide que el atributo más adecuado es Fuerza y que la
dificultad es 3. El jugador mira a su hoja de personaje y comprueba que su fuerza
es 18. Ahora lanza los dados obteniendo un total de 17 en los dados, sumando la
dificultad tiene 20 y restando su nivel llega a 19, que es mayor que su atributo
de fuerza y, por ello, ha fracasado.

Recuerda: Dados + Dificultad - Nivel < Atributo


\section{Salvar}

En las pruebas de salvar o tiradas de salvación tratas de que tu personaje
escape a un efecto adverso. Se trata de una reacción que el jugador no la anunciado,
pero obvia. Por ejemplo, si entra en una trampa y le va a caer una piedra encima
debe Salvar por Destreza para esquivarla.

Las pruebas de salvar se hacen con una prueba de atributo, con la dificultad que
el máster estime en cada caso. La Destreza suele ser el atributo más adecuado para
aquello que implique movimientos rápidos, como saltar o agarrarse a un precipicio
en el último segundo. La Inteligencia se suele emplear para no dejarse engañar por
un espejismo mágico. La Sabiduría viene bien para resistir magia manipulativa. Por
último la Constitución es, normalmente, la más adecuada para resistir venenos y
enfermedades.

\minisec{Fallos seguros}

Una tirada de 5 o menos en un Prueba o Salvación es siempre un fallo.

\minisec{Pruebas y Salvaciones de Monstruos}

Los monstruos hacen las pruebas contra su nivel de monstruo más 10. Así, un
goblin de nivel 1 tendría un 11 en todas sus atributos, a efectos de pruebas y
salvaciones.

Sin embargo el máster puede tener en cuenta las especiales características de
cada monstruos cosa. (Por ejemplo puedes decir que un zombi salva por DES con un
valor de solo 7, por lo torpes que son.)

\section{Iniciativa}

En cada turno de combate, cada grupo de luchadores, tira $ 1d20$ por iniciativa. El
grupo que saque más actúa primero, repitiéndose los empates.

La única excepción es cuando un grupo ha sorprendido al otro. En ese caso el grupo
que ha sido sorprendido no actúa en absoluto durante ese turno.

\section{Niveles}

En \enquote{Laberintos Tenebrosos} todas las cosas, personas, monstruos y objetos
tienen niveles. Los niveles son una forma aproximada de indicar cuál es su capacidad.
Lo más mundano y simple tiene nivel 0, pero la mayoría de los
elementos a los que se enfrentarán los personajes jugadores tienen nivel 1 o superior.

En general una cosa o monstruo tendrá un nivel aproximadamente igual o inferior
al del Laberinto en que se encuentre. Por ejemplo en el nivel 3 de un laberinto,
---o en un castillo, o incluso un bosque encantado de nivel 3---, la mayoría de
las pociones que se encuentren serán de nivel 3. Serán de nivel 3 también muchos
de los monstruos a los que se enfrenten.

Una excepción son los tesoros. Los tesoros de nivel 1 a 6 están diseñados para
representar lo que un individuo puede llevar encima mientras que los niveles 7 y
superior para lo que pueda contener las \enquote{cámaras del tesoro} de un laberinto.

\section{Ataques}

Para realizar un ataque tira un $ 1d20$ y suma (o resta) el modificador por el
atributo adecuado (Fuerza si son armas de cuerpo a cuerpo, ---como una espada---,
y Destreza si son armas a distancia, ---como un arco). El número que tienes que
sacar debe ser mayor que el resultado restar la Clase de Armadura del Objetivo a
tu Capacidad de Ataque.

$1d20$ + Modificador > Capacidad de Ataque - Clase de Armadura

\minisec{Ejemplos de Ataque}

Eso suena muy difícil, así que pongamos un ejemplo. Eres un guerrero de nivel 1
y atacas a un Kobold con tu espada. Tu Fuerza es 13, así que tienes un
modificador de +1. Tiras $ 1d20$, sacas un 16 y le sumas tu modificador. Así que tu
total es 17.

Ahora, como eres un guerrero de nivel 1, tu Capacidad de ataque es 19. El kobold
tiene una Clase de Armadura de 6. Restas a tu Capacidad de Ataque su Clase de
Armadura y tienes un 19-6 = 13.

Y comparamos 17 es mayor que 13 por lo que le has dado al kobold.

\begin{itemize}
\item Fuerza: 13 +1
\item Capacidad de Ataque: 19
\item Armadura del Objetivo: 6
\end{itemize}

$\textbf{16} + 1 > 19 - 6$ ¡Impacto!

Supongamos que sacaras un \textbf{12}. La cosa cambia. Ahora tu total es de
\textbf{12} + 1, 13, que es exactamente lo mismo que resta 19 - 6. El resultado
no supera lo que tenías que sacar, 13 y, por tanto, fallas.

$\textbf{12} + 1 = 19 - 6$ ¡Fallo!

\minisec{Ataques a Distancia}

Añade 1 punto a la clase de Armadura de tu objetivo por cada 30 metros completos
de distancia que estén alejados de tu objetivo. Añade 1 punto más si hay lluvia.
Por último añade 3 puntos si está medio protegido por un muro o un obstáculo
semejante, siempre que asome algo que se pueda golpear.

\begin{table*}[p]
\centering
\caption{Capacidad de Ataque}
\begin{tabular}{lcccccc}
\toprule
Clase           & 1 & 2 & 3 & 4 & 5 & 6\\
\midrule
\midrule
Humano Normal   & 20\\
Ladrón          & 19 & 19 & 19 & 18 & 18 & 17\\
Guerrero        & 19 & 18 & 17 & 16 & 15 & 14\\
Mago            & 20 & 19 & 18 & 17 & 17 & 16\\
Otros           & 19 & 19 & 18 & 17 & 16 & 15\\
Monstruos       & 19 & 19 & 18 & 18 & 17 & 17\\
\bottomrule
\end{tabular}

\end{table*}

\minisec{Pifias en ataque}

Si sacas un 1 en una tirada de ataque, puedes haberla pifiado. Para comprobarlo,
haz una prueba de Destreza. Si la fallas habrás pifiado y tu personaje se caerá
al suelo.

Sin embargo, si sacaras un triple uno al hacer la prueba de destreza, la pifia
será una pifia trágica y tu arma habrá herido al compañero que tengas más cercano
o, si eso no es posible, a ti mismo.

\minisec{Aumentar o Disminuir la Clase de Armadura}

A veces el máster puede decidir bajar temporalmente la clase de armadura normal
de un personaje o monstruo para reflejar circunstancias especiales, como un
personaje que tiene las manos atadas o un monstruo apenas visible en la
oscuridad.

\includegraphics[width=\columnwidth]{lira.png}

\section{Daño y Muerte}

Para determinar el daño, comprueba el daño que hace tu arma y suma (o resta) tu
modificador por Fuerza. El resultado es el número de puntos que el máster
restará a la vida del monstruo.

Si los puntos de vida de un monstruo llegan a 0, habrá muerto y ganarás sus puntos de
experiencia.

Si puntos de vida de un personaje jugador llegan a 0 cae inconsciente y no puede
hacer nada más, ni siquiera hablar, hasta que termine el combate. Un personaje
inconsciente puede ser rematado por un monstruo que esté a su lado, pero, por lo
general, los monstruos preferirán luchar contra los enemigos que todavía les
pueden hacer daño.

Si todos los personajes quedan inconscientes quedarán a merced de los monstruos
quienes normalmente los rematarán o los convertirán en esclavos, robándoles todo
su equipo, hasta sus zapatos, y llevándolos a su base o vendiéndolos a un esclavista.

Terminado el combate un personaje inconsciente se recupera y gana 1 punto de
vida.

\minisec{Hambre}

Las reglas de hambre son simples. Cada día en el que el personaje no coma una
ración completa el personaje ha de Salvar Constitución. La dificultad es de 1
si no ha comido nada en absoluto y 1 más por cada tres días sin comer. Si falla
la tirada de salvación baja 1 punto, al azar, entre uno de sus atributos.

Si alguno de los atributos llega a 0 el personaje es incapaz de moverse. Si dos
de los atributos llegan a 0 el personaje muere.

\minisec{Enfermedades y Venenos}

Es imposible crear reglas para todos los enfermedades y venenos posibles. En
general el personaje ha de salvar Constitución. Si falla sufrirá las consecuencias
que dependerán de cada veneno o enfermedad. Lo más simple es que pierda un número
de puntos de daño, ---por ejemplo $1d6$---, o de algún atributo.

\minisec{Caídas}

Las caídas de menos de tres metros no producen daño alguno en
\enquote{Laberintos Tenebrosos}. Sin embargo, a partir de ahí, hacen $1d4$ de daño
por cada tres metros completos. (O sea $2d4$ a los seis metros, $3d4$ a los nueve metros y
así). El dado es de $1d6$ si la caída es sobre suelo de piedra o similar.

Si caes sobre estacas afiladas añade, además, $1d6$ extra.

\minisec{Fuego}

El daño del fuego es por turno que tu personaje se exponga a él. En el caso de
una antorcha será de $1d6$ y sigue subiendo hasta 6d6. Un caso extremo es caer
sobre lava. Primero sufres el daño de la caída como si hubieras caido sobre
piedras y, después, $6d6$ de daño por fuego y turno.

\section{Curación}

Los personajes pueden recuperar puntos de vida mediante conjuros, pociones
mágicas y la habilidad de curar del paladín. Esta recuperación nunca puede superar el que tenían al
comenzar la aventura (o al subir de nivel).

También se recuperan puntos de vida descansando. Una noche de descanso se recuperan $1d4$ puntos
de vida. Esta recuperación tampoco puede superar el número de puntos de vida que tenía al
comenzar la aventura (o al subir de nivel).

\section{Experiencia}

Se ganan puntos de experiencia de tres maneras distintas: Derrotando a monstruos
(no sólo matarlos), resolviendo aventuras y encontrando tesoros.

Los mostruos dan, si no se dice otra cosa, tantos puntos de experiencia como su nivel
multiplicado por 5.

Cuando el máster considere que los jugadores han resuelto una aventura, o vencido
a un monstruo principal o sobrevivido a un laberinto, les debe recompensar con
1000 a 2000 puntos por nivel de la aventura. Estos puntos deben repartirse entre
todos los personajes jugadores.

En general se gana 1 punto de experiencia por moneda de oro que encuentre cada jugador.
Otros tipos de tesoro dan tantos puntos de experiencia como monedas de oro cuesten.

\section{Subir de Nivel}

Cuando se acumulen los suficientes puntos de experiencia (XP), se pasa de nivel. Esto
sucede inmediatamente pero nunca en medio de un combate.

Al subir de nivel se producen los siguientes beneficios.

\begin{enumerate}
\item Se recuperan todos los puntos de vida perdidos.
\item Se lanza 1 dado de puntos de vida, se añade el modificador por Constitución y se suma un punto extra. En cualquier caso, se gana como mínimo 1 punto.
\item Se lanza 1$  d20$ por atributo. Si la tirada es menor que el atributo, añade un punto a tu atributo.
\item Los magos y elfos \emph{pueden ahora aprender} 1 conjuro nuevo. Para ello deben encontrar un maestro o un libro de conjuros y pasar al menos un día estudiando el nuevo conjuro. A terminar ese periodo hacen una prueba por Inteligencia añadiendo como dificultad el nivel de conjuro.
\item Los paladines \emph{aprenden} un nuevo conjuro haciendo una peregrinación
a un templo.
\end{enumerate}

\begin{table*}[p]
\centering
\caption{Experiencia para subir de nivel}
\begin{tabular}{lccccc}
\toprule
Clases  & 2 & 3 & 4 & 5 & 6\\
\midrule
Guerrero & 2,035 & 4,065 & 8,125 & 16,251 & 32,501\\
Paladín & 3,500 & 5,251 & 12,251 & 22,501 & 45,001\\
Mago & 2,500 & 5001 & 10,001 & 20,001 & 40,001\\
Ladrón & 1,251 & 2,501 & 5,001 & 10,001 & 20,001\\
Elfo & 4,065 & 8,125 & 16,251 & 32,501 & 65,001\\
Enano & 2,101 & 4,201 & 8,751 & 17,501 & 35,501\\
Hobbit & 2,305 & 4,605 & 8,125 & 16,251 & 32,501\\
\bottomrule
\end{tabular}

\end{table*}

\section{Carga}

Un personaje puede transportar hasta 12 objetos en su mochila y hasta tres armas
del tipo de las espada, hachas o maza colgando de su cinto. Puede llevar también
un escudo en un brazo o fijado a la espalda. El arco puede llevarse también a la
espalda si no se lleva escudo. La honda no ocupa espacio alguno y la bolsa de
piedras puede llevarse al cinto.

Además de eso un personaje puede transportar tantos kilos como su $Fuerza + 6$, sin
perjuicio. Si lleva más de eso sufren un perjuicio de un punto a todas sus tiradas.

En ningún caso pueden llevar más del doble de kilos que su Fuerza.

\section{Esclavitud}\label{s:esclavitud}

Un personaje puede ser sometido a la esclavitud si es esclavizado por un brujo
goblin u otro tipo de criatura malvada con esos poderes. Para
esclavizar a un personaje el monstruo debe rodear su cuello con una argolla
metálica y pintar o grabar un símbolo en su piel. Esta ceremonia es mágica, no
puede resisitirse, y supone las siguientes consecuencias al desventurado esclavo:

\begin{enumerate}[label={\alph*)}]

\item Pierde todos sus conjuros y es incapaz de ejercer la magia, con la excepción de conjuros de curación y luz.

\item Los elfos y los enanos pierden su capacidad de infravisión.

\item La capacidad de ataque del personaje se fija en 20.

\item Los ladrones pierden su capacidad de abrir cerraduras.

\item Su piel se vuelve azul.

\item Un personaje así esclavizado gana 1 punto de experiencia por cada día que
permanezca como esclavo. Gana 10 puntos por cada orden que cumplan a la total
satisfacción de su amo.

\end{enumerate}

\minisec{Liberarse de la esclavitud}

Un personaje se libra de la maldición de la esclavitud cuando consigue estar
en libertad durante dos días seguidos.

\section{Moral}

La mayoría de los monstruos no lucharán hasta la muerte si es evidente que van a
ser derrotados. En esas circunstancias el máster tira $ 1d20$ y si el resultado
supera el valor de moral de los monstruos supervivientes, éstos tratarán de huir.

Si no es posible se rendirán si tienen la inteligencia para ello. Si no, atacarán
tratando de abrirse paso a la salida más cercana.

El valor de moral de la mayoría de los monstruos es de 5 + su nivel.

\vspace{1pc}
\includegraphics[width=\columnwidth]{06.jpg}

\section{Magia}

Elfos, Magos y Paladines tienen la habilidad de lanzar los conjuros que tengan
aprendidos. Solo pueden lanzar estos conjuros un máximo de una vez por día,
recuperando la capacidad con cada nuevo amanecer.

\section{Conjuros de Mago y Elfo}

\begin{description}
\item[Familiar] Permite al mago trabar amistad con un animal, nunca mayor que una lechuza o un perro medio. Dicho animal le será siempre fiel y entendrá todas sus instrucciones. Este conjuro solo se puede lanzar sobre un animal que ya sea amigo del mago. Solo se puede tener un familiar a la vez. Un familiar no puede regalarse ni venderse. Tampoco puede el mago crear familiares para otras personas.
\item[Luz] Crea una luz en un objeto que ilumina como una vela. 1 Hora.
\item[Encantar] Encanta a criaturas que totalicen $1d3$ niveles por cada 2 niveles del mago. El mago puede dar una orden que la criatura o criaturas cumplirán.
\item[Proyectil Mágico] Lanza un proyectil mágico que impacta en el objetivo sin remisión haciendo, por lejos que esté siempre que el mago lo pueda ver, $1d4$ puntos de daño por nivel del mago.
\item[Escudo] El mago resta 1 punto a su Clase de Armadura por nivel. También puede lanzarse sobre otra persona.
\item[Dormir] Duerme a criaturas que totalicen $1d6$ niveles por nivel del mago.
\item[Oscuridad] Crea una oscuridad que bloquea todo tipo de visión, incluso infravisión, en un área de 20 metros cuadrados. 1 hora.
\item[Invisibilidad] Una criatura que pueda tocar (o él mismo) se vuelve invisible hasta que haga un ataque.
\item[Toc, toc] Abre una cerradura.
\item[Levitar] El mago flota en el aire hasta 2 metros. 10 minutos por nivel.
\item[Quitar Magia] Hace desvanecerse los efectos de un conjuros.
\item[Ilusión] Crea una ilusión que parece real, pero desaparece al contacto. El tamaño de la ilusión depende del nivel del mago. A nivel 1 puede crear algo del tamaño de una moneda, a nivel 2 de un cofre, a nivel 3 de un niño, a nivel 4 de un adulto, a nivel 5 de un caballo y a nivel 6 de una casa. 1 hora.
\item[Muro] Crea un muro de piedra de hasta 3 metros de alto, 20 metros de largo y 1 metro de ancho. 1 hora.
\item[Resbalo] El objetivo, que el mago debe poder ver, debe salvar DES cada turno o caer al suelo. Duración 15 minutos.
\item[Empatía] El mago puede adivinar las emociones y pensamientos de todas los monstruos
que estén a su vista. Duración 1 turno.
\item[Lluvia de meteoros] El mago designa un círculo con un radio de su nivel en metros, o inferior si así lo desea. Dentro de ese círculo caerán una lluvia de meteoros que hace 1 puntos de daño por cada dos niveles del mago.
\end{description}

\section{Conjuros de Paladín}

\begin{description}
\item[Curar Heridas] Cura $1d4$ Puntos de vida por nivel del paladín.
\item[Detectar el mal] Todo lo maligno que esté a $1d4$ metros por nivel del paladín brilla 5 minutos.
\item[Luz] Crea una luz en un objeto que ilumina como una vela. 1 Hora.
\item[Purificar Comida y Bebida] Quita las enfermedades que pudieran tener.
\item[Bendecir] Los aliados ganan +1 a la tirada de ataque y -1 a las pruebas.
\item[Encontrar Trampas:] Encuentra todas las trampas a $ 1d4$ metros por nivel del paladín. Dura 10 minutos.
\item[Paralizar] Paraliza criaturas y personas que totalicen $ 1d4$ niveles por cada 2 niveles del paladín. Dura 10 minutos.
\item[Hablar con Animales] Puede hablar y entender a los animales
\item[Curar Enfermedad] Cura una enfermedad a una persona o criatura que pueda tocar.
\item[Quitar maldición] Quita una maldición, como la esclavitud, a una persona u objeto.
\item[Hablar con los muertos] Permite hacer tres preguntas a un muerto, cuya tumba esté a menos de 1 metro.
\item[Dispersar muertos vivientes] Hace que huyan no muertes que totalicen $ 1d3$ niveles por cada 2 niveles del paladín. 10 minutos.
\item[Sacrificio] El paladín sacrificia su propia salud por otra persona. Pierde 1d6 puntos por nivel del paladín y, a cambio, restaura esos mismos puntos de vida a otra persona o criatura o cura de cualquier enfermedad o maldición. El paladín puede luego recuperarse por medios normales. Si el paladín siempre se quedará con  al menos 1 punto de vida.
\item[Lenguas] Permite al paladín entender y hablar en cualquier idioma. 15 minutos.
\item[Bendecir agua] Permite al paladín bendecir suficiente agua para 1 uso. Este agua no debe mezclarse.
\end{description}

\section{Conjuros de Brujo}

Estos conjuros son típicos de los goblins y otros monstruos. No obstante, y siempre
a juicio del máster, los monstruos pueden disponer de otros conjuros.

\begin{description}
\item[Fétido] El brujo crea una nube fétida de $1d12$ metros de radio cuyo centro queda a una distancia de hasta 12 metros. El objetivo salva CON o perjuicio de 1 a todas las tiradas.
\item[Guerra] El brujo lige una criatura, amiga o enemiga que pueda ver. Esa criatura se vuelve loca, corre al enemigo más cercano y ataca sin preocuparse por sí misma. Beneficio de 2 a la capacidad de ataque, +2 al a daño, pero +2 a la clase de armadura. Si se lanza sobre un enemigo ---para llevarlo a una pelea que no puede ganar---, éste puede resistirse si salva SAB.
\item[Fuego] Lanza una bola de fuego de medio metro de radio. El brujo elige un punto, y tira para ver cuánto se desvía. La bola de fuego aparece a $1d6-1d6$ metros al norte y $1d6-1d6$ metros al este.  Las criauras que estén dentro de la bola (medio metro del punto) sufren $3d4$ de daño. $2d4$ hasta un metro, $1d4$ hasta dos metros y nada más allá de dos metros.
\item[Destrucción] El brujo elige un elemento de piedra, ladrillo, granito o mármol a una distancia de 20 metros o menos. El elemento se destruye de forma explosiva, creando una brecha de 1 metro de ancho por tres de alto y tres de profundidad. Las criaturas que estén a menos de tres metros, y no salven DES sufren $1d6$ puntos de daño por los cascotes que salen volando en todas direcciones.
\item[Hierro Ardiente] El brujo elige un objeto de hierro que pueda ver. Dicho objeto queda bajo una maldición que consiste en que se calienta tanto que no puede sostenerse con las manos. EL objetivo debe salvar CON o elegir entre desprenderse del objeto o sufrir 1d6 puntos de daño. No puede lanzarse sobre objetos mágicos. Duración: 1 combate.
\item[Esclavitud] Ver reglas de Esclavidud en la página \pageref{s:esclavitud}.
\item[Podredumbre] A distancia, 20 metros. Todo objeto que lleve encima que se pueda pudrir, como la comida y la bebida, se pudre y no puede usarse. No afecta al agua bendita.
\item[Olvido] El brujo lanza el conjuro sobre un objetivo que pueda ver. Si el objetivo no salva Sabiduría todos los que le conozcan se olvidarán de su existencia. Este conjuro se suele usar para preparar secuestros y otros crímines. Duración 1 año.
\item[Tormenta] El brujo provoca una tormenta, que puede comenzar incluso dentro de una habitación y puede extenderse a un área del tamaño y forma que el brujo desee hasta un círculo de 1 kilómetro de radio. El primer turno se levanta una brisa. El segundo turno el viento es tan fuerte que cierra todas la puertas e impide que se puedan usar armas a distancia y llueve, apagando antorchas. El tercer turno todas las criaturas dentro del área de la tormenta, que no estén a resguardo deben salvar DES o sufrir 1d4 puntos de daño por impactos de pequeños objetos. Además no pueden moverse. La lluvia aumenta la intensidad. En el cuarto turno la tormenta desaparece.
\item[Nube de calamar] El brujo genera una nube negra que impide todo tipo de visión, incluso mágica. Dicha nube es de tres metros de radio y puede
colocarse a una distancia de hasta 12 metros del brujo. Quien intente entrar en esa nube queda manchado de una tinta negra y espesa, que no puede lavarse sino tras 1 hora de esfuerzo, jabón y agua caliente.
\item[Poseer] El brujo debe tocar a una criatura que permanezca quieta. Si dicha criatura no salva por SAB o por nivel, ---bien porque no se resista o falle la tirada---, quedará poseida por el brujo. El antiguo cuerpo del brujo muere al instante. El brujo adquirirá todas las competencias de la criatura y, además, podrá lanzar sus conjuros como hasta ahora.
\item[Miedo] Distancia 20 metros. Duración 1d4 turnos. Salva SAB cada turno o huye del combate, arrojando tus armas al suelo y corriendo en la dirección opuesta a tus enemigos.
\item[Agujero Negro] Crea un agujero negro en el suelo de 50 centímetros de radio a una distancia de hasta 20 metros del brujo y que dura un combate. Cualquier criatura que caiga en su interior desaparece para siempre.
\end{description}
\end{multicols}

\clearpage

\part{Laberintos}

\begin{abstract}
Llamamos \enquote{laberinto}, ---en realidad---, a cualquier lugar de aventuras.
Un laberinto podría ser una cueva, un castillo, un bosque o incluso una nave
espacial abandonada hace siglos por visitantes alienígenas a tu mundo de fantasía
medieval. Sin embargo, la forma más común de \enquote{laberinto} es, bueno, un
laberinto subterráneo lleno de pasillos estrechos y serpenteantes, habitaciones,
cámaras secretas, celdas, rejas, trampas, escaleras y pasadizos secretos. En su
interior residen una extraña variedad de habitantes: animales de la oscuridad,
monstruos, goblins, sus esclavos, y todo lo que el máster quiera añadir.
\end{abstract}
\vspace{1cm}
\begin{multicols}{2}

\includegraphics[width=\columnwidth]{dungeon.png}

\section{Tu propio laberinto}

El máster deberá proporcionar a los jugadores un \enquote{laberinto}. Afortunadamente,
hoy hay miles de \enquote{dungeons} ---como se denominan en Inglés--- disponibles
para descargar. Incluso hay herramientas, como https://donjon.bin.sh/ que generan
dungeons en segundos.

Pero si el máster quiere crear un dungeon, le basta un papel cuadriculado. Que
tu primer dungeon sea pequeño, pon doce habitaciones, en una pon el tesoro, en
otra pon celdas, pon también cocina, comedor y baño. Añade, si quieres, una sala
del trono para el caudillo de los goblins o el monstruo principal. Decide qué vas
a poner en las otras habitaciones a tu gusto. Por último por una entrada principal y,
quizás, una entrada alternativa.

Ahora puebla el dungeon con habitantes, tesoros y trampas. Recuerda el nivel de tus personaje y no
se lo pongamos ni demasiado difícil ni demasiado fácil.



\section{El mundo}

\enquote{Laberintos Tenebrosos} transcurre en un mundo de juego de fantasía medieval.
Eso significa que se inspira en la edad media europea, pero añadiendo tanta fantasía
como quieras. Cada máster es responsable de su propio mundo de juego, pero mi consejo
es que empieces por un lugar pequeño. Todo lo que hace falta es una base para los
jugadores, unas tierras salvajes y, en medio de ellas, un laberinto. ¿Qué es puede
ser una base? Un pueblo, una ciudadela, un castillo o incluso un barco desde donde
los jugadores salen a investigar el mundo y vivir aventuras mientras el resto del
mundo siguen con sus vidas. Dibuja un mapa con estos elementos, añade unos pocos
detalles más y quédate contento con esto. Ya irás añadiendo más según progresen
las aventuras de los jugadores. Con el tiempo acabarás teninedo tu propio mundo
único.

También puedes jugar en un mundo de fantasía medieval que conozcas de una novela
o una película. O que te inspires en ellos para crear tu propia versión.

\section{Monstruos}

Un monstruo es, de forma genérica, toda criatura que vive en un laberinto. Algunos
son hostiles a los jugadores e intentarán matarlos, capturarlos, comerlos o hacerles
algo mucho peor. Otros pueden ser indiferentes a los jugadores y solo reaccionarán
si los jugadores interactúan con ellos. Incluso podrán encontrarse gente amiga, como
los prisioneros que quieran recuperar la libertad.

Hay un cuadro de monstruos en la página \pageref{c:monstruos}, que representa a
algunos de los monstruos más comunes en los laberintos, ordenados por nivel. El
máster puede, por supuesto, añadir nuevos monstruos a esta cuadro o inventarse los
suyos propios. En ese caso intenta que tus monstruos no sea simplemente máquinas
de matar. Añade habilidades divertidas o únicas, dales alguna historia, quizás
alguna vulnerabilidad y algo que justifique que son como son.

Cuando dirijas la partida sé lógico. Recuerda que es raro que un zombi tenga algún
tesoro encima; que los animales nunca buscarán hacer prisioneros, sino convertilos
en comida y que un cubo gelatinoso no va a negociar con nadie.

Ten en cuenta también el nivel del laberinto, que la mayoría de tus monstruos sean
de su mismo nivel. Pon algunos inferiores y, si quieres, un par superiores que sirvan
de jefes o de objetivo final.

\includegraphics[width=\columnwidth]{hut.png}

\begin{table*}[p]
\caption{Monstruos, monstruos, monstruos}
\label{c:monstruos}
\centering
\begin{threeparttable}
\begin{tabular}{lcccp{8cm}}
\toprule
Criatura             & Nivel &   CA & Daño & Notas\\
\midrule
Escarabajo de Fuego  &  1 &  6  & Deslumbra &  Impacto automático. Tirada de CON o incapaz de ver $ 1d6$ minutos\\
Murciélago Vampiro   & 1 & 6 & $ 1d4$ & Salva CON el turno siguiente o Pierdes $ 1d6$ de vida extra.\\
Ciempiés gigante    & 1 & 7 & 0 & Salva CON o quedas insconsciente hasta terminar el combate.\\
Demonio Manes       & 1 & 7 & $ 1d4$ & Recibe mitad del daño de armas no mágicas. \\
Goblin verde & 1 & 6 & Arma & Solo $ 1d6$ vida\\
Goblin azul & 1 & 9 & Arma & Solo $ 1d6$ vida si son goblins verdaderos\\
Zombi & 1 & 9 & $ 1d4$ & Salva CON o enfermedad, zombi en $ 1d6$ días.\\
\midrule
Esqueleto & 2 & 7 & Arma & \\
Goblin negro & 2 & 5 & Arma & \\
Hormiga gigante & 2 & 6 & $ 1d6$ & Veneno. Salva CON o sufre $ 2d6$ más de daño.\\
Ghoul & 2 & 7 & $ 2d3$ & Salva CON o Paralizado.\\
\midrule
Arpía & 3 & 6 & $ 2d3$ & Vuela. Canto: Salva CAR o debes caminar a dónde te mande.\\
Hombre rata & 3 & 5 & $ 2d4$ & No pueden ser sorprendidos. \\
Sombra & 3 & 7 & $ 1d4$ & -1 a Fuerza, solo se pueden herir con armas mágicas\\
Cambiaformas & 3 & 6 & $ 2d4$ & Puede cambiar de forma a voluntad\\
Goblin brujo & 3 & 5 & Arma & Magia: Esclavitud, Fétido, 1 conjuro de mago\\
Fantasma & 3 & 7 & Chupa 1 nivel & Sólo puede herirse con armas mágicas\\
Orco & 3 & 4 & Arma + 1 & \\
\midrule
Oso grizzly & 4 & 5 & $d6$ & Tres ataques por turno$^1$.\\
Gárgola & 4 & 4 &$  d6$ & Tres ataques por turno$^1$.\\
Cubo Gelatinoso & 4 & 2 &$  d10$ & Salva CON o paralizado. Inmune al frío y rayos.\\
Hombre Lobo & 4 & 5 &$  d10$ & Solo puede atacarse con armas de plata.\\
\midrule
Owlbear & 5 & 4 &$  d10$ & Cuatro ataques por turno††.\\
Basilisco & 6 & 2 &$  d8$ & Si te mira a los ojos, salva CON o petrificado\\
Momia & 6 & 4 & $ 2d6$ & Las heridas solo se curan con magia. Inmune a las armas normales. Las armas mágicas solo le hacen la mitad de daño.\\
Demonio de Balor & 9 & 1 &$  d12$ + 2 & Ataque de Látigo$^2$. Salva DES o te acercará a sus llamas, $ 3d6$ de daño por fuego\\
Dragón &11 & -1 & $ 1d8$ & Aliento de Fuego: Alcanza a $ 1d4$+2 objetivos a 20 metros
del dragón ($ 3d8$ daño). Conoce $ 2d3$ conjuros. Vuela\\
\bottomrule
\end{tabular}
\begin{tablenotes}
\item 3 Cuando falle un ataque pierde el resto de los ataques.
\item 4 Puede hacer un ataque normal y otro de látigo contra el mismo o distinto enemigo.
\end{tablenotes}
\end{threeparttable}
\end{table*}



\section{Tesoros}

Todos los laberintos y algunos de sus habitantes están cargados de tesoros, aunque
la mayoría de las veces no es fácil saquearlos, ---debiendo sortearse sortilegios,
puertas secretas, candados, trampas mortales, acertijos imposibles y otros peligros.
Sin embargo, para los triunfadores, las recompensas pueden ser magníficas e incluyen
objetos mágicos.

Los tesoros que contenga un laberinto quedan al criterio del máster. El máster decide
qué tesoros hay, cuánto hay y dónde y cómo están escondidos. No obstante, el máster
\emph{puede} usar los cuadros de tesoros que reflejan, más o menos, lo que normalmente se suele hacer.

\section{Objetos mágicos}

\minisec{Objetos malditos}

Algunos objetos, como el anillo de Frodo, pueden estar malditos. Un objeto madilto
no es otra cosa que un objeto con una maldición que afecta adversamente a los
personajes. Sin embargo, los más interesantes son aquellos que ofrecen también un
beneficio.

No hay objetos malditos en los cuadros, sino que el máster debe dejarlos sabiamente
en la aventura. Quizás advirtiendo antes de su existencia o solo dando pistas, alertando
y amenazando con sus poderes.

Ejemplo de objeto maldito aburrido: Espada -1. Esta espada da un -1 a la tirada de
ataque, además se queda pegada a la mano del personaje.

Ejemplo de objeto maldito interesante: La Armadura de la Sangre. Se trata de una
armadura de placas siempre ensangrentada de clase de armadura -1 y que reduce el
daño de todas las armas a la mitad. Sin embargo, provoca el odio de todas los monstruos
haciendo que estos siempre ataquen a su portador.

\minisec{Pociones}

Las pociones vienen en pequeños frascos y no son identificables por los jugadores
si no vienen etiquetados, ---lo que casi nunca ocurre. La mayoría son beneficiosas
pero muchas son venenos o suponen efectos perversos, como la esclavitud.

La mayoría de los venenos simplemente quitan $ 1d6$ puntos de daño por nivel de la
poción. Los efectos de los venenos pueden ser instantáneos o progresivos, avanzando
poco a poco según se pierdan las Salvaciones de Constitución.

Ejemplo: Poción de Esclavitud. Si el personaje no salva CON le aparece un punto azul
en la piel. Este punto se extenderá en una hora a una extremidad, y, progresivamente
hasta teñir todo el cuerpo en $ 3d4$ horas. Cuando eso ocurra habrá caído bajo la maldición
de la esclavitud.

Ejemplo: Poción de Debilidad. Si el personaje no salva CON pierde $ 1d4$ de CON y $ 1d4$ de Fuerza.
Si quedara a 0 muere.

\minisec{Agua bendita}

El agua bendita es especial y puede usarse de modo creativo, siempre respetando
su origen sagrado. En cualquier caso la decisión del máster sobre esto es definitiva.
Sin embargo, a mí me gusta usarla para lo siguiente:

\begin{enumerate}[label={\alph*)}]
\item Quitar algunas maldiciones
\item Curar enfermedades causadas por demonios
\item Convertir a una criatura. En este caso debe ser una criatura que ya quiera hacerse \emph{buena}, aunque por naturaleza no lo sea, como un orco.
\item Recuperar 1 punto de vida.
\item Ácido contra los demonios. Para un demonio el agua bendita es como ácido, causándole 4d4 puntos de daño.
\end{enumerate}

Un frasco de agua bendita genera suficiente agua para un uso.

\minisec{Anillos}

Un anillo da a su portador efectos permanentes mientras ponga el anillo en el
dedo anular de su mano preferida. Casi siempre otorga los efectos de un conjuro
o una mejora en una habilidad.

Para usar un anillo debe ponerse en el dedo anular de la mano preferida (derecha
o izquierda). Cualquiera puede ponerse y usar un anillo.

\minisec{Papiros}

Los papiros, o pergaminos, son rollos con conjuros mágicos que se activan leyendo
sus palabras en voz alta. Puede usarlos cualquiera,
pero solo un mago ---no un elfo, ni un paladín--- puede entender su significado. Es
decir, si no hay un mago los personajes tendrán que usar el papiro a ciegas.

El pergamino es de un solo uso, y las letras del conjuro se borran en cuanto se han leído.

\minisec{Varitas}

Una varita contiene poderes conjuros que solo puede usar un mago o un elfo ---no un paladín---
y permite lanzar los conjuros que contenga, que suele ser $ 1d4$ por nivel de la varita. Una vez
agotados la varita queda descargada.

\minisec{Armas}

Muchas armas pueden ser mágicas. A veces no contienen beneficios especiales, pero el hecho
de que sea mágica significa que puede usarse contra monstruos a los que no les afectan las
armas normales, como los fantasmas.

Los efectos más comunes es que añadan de +1 a +3 a la capacidad de ataque o que dañen de
forma más grave a determinados monstruos. Como \enquote{Rebanorcos}, una daga que añade $ 3d6$ de daño cuando lucha contra orcos.

\minisec{Armaduras}

Las armaduras pueden encantarse y, normalmente, tienen clases de armaduras muy bajas,
incluso negativas, o tienen efectos especiales como añadir +1 de FUE a su portador, o
les permitan nadar con ellas puestas.

\minisec{Objetos mágicos únicos}

El máster puede crear sus objetos únicos que se salgan de estas reglas. No obstante,
cuando salgan en un cuadro de tesoros, lo normal es que interprete que se trata de un
objeto con los efectos de un anillo o una varita, pero de otra forma. Por ejemplo, la
\enquote{Corona de Sabiduría} que añade +1 a la Sabiduría del que se la ponga.

El máster puede inspirarse en películas, novelas, juegos o series de televisión
para crear sus objetos o puede sacarlos de su imaginación. Pero debe cuidar de que
no sean excesivamente poderosos o que, si lo son, estén malditos.

Ejemplo de objeto único. Goblin chillón de piedra. Se trata de una estatuilla de
piedra de 20 centímetros de alto que se pone en cualquier lugar que desee vigilarse.
Se quita la venda al Goblin chillón y, a partir de entonces, cada vez que alguien
pase delante el Goblin chillón gritará sin parar con un sonido que puede escucharse
a un kilómetro o a 10 con la habilidad de Susurros.

\begin{table*}[p]
\caption{Tesoros}
\centering
\begin{threeparttable}
\begin{tabular}{ccccccc}
\toprule
Clase & mc  & mp  & mo  & mP  & Piedras preciosas & Magia\\
\midrule
1 & $ 4d6$ \\
2 & & $ 2d8$+1\\
3 & & & $ 1d10$+1\\
4 & & & & $ 1d8$\\
5 & & & & & $ 1d6$\\
6 & $ 3d20$ & $ 1d20$ & $ 1d20$ & $ 1d4$  & $ 1d4$ - $ 1d4$  & †(2 - $ 1d8$)\\
7 &  & $ 1d20$ & $ 2d20$ & $ 1d6$ & $ 1d6$ - $ 1d4$ & †(2 - $ 1d6$)\\
8 &  &  &  & & &  $ 1d4$ - 2 pergaminos\\
9 &  &  &  & & &  $ 1d4$ pociones\\
10 &  &  & $ 2d4$ & & $ 4d4$  \\
11 &  &  &  & & $ 1d3$  \\
12 &  & $ 1d2$  & $ 1d4$-$ 1d4$   \\
13 & $ 1d6$ - $ 1d4$ & $ 1d4$ - $ 1d4$  \\
14 &   &  &  & $ 2d4$ - $ 1d6$ & $ 2d6$  & †(1) \\
15 & $ 2d4$ & $ 3d20$ & $ 3d20$ & $ 2d4$ & $ 2d20$& †(3)\\
16 &  &  &   $ 3d4$ & $ 2d4$x7 & $ 4d20$ & †(6)\\
\bottomrule
\end{tabular}
\begin{tablenotes}
\item Desde la Clase 10 las monedas encontradas se multiplican por 1000, excepto las de platino, que se multiplican por 100
\item † tirar en el cuadro de Tesoros mágicos tantas veces como aparezca el número entre paréntesis
\item Si el resultado de los dados da negativo, se entiende que es un cero.
\item
\end{tablenotes}
\end{threeparttable}
\end{table*}

\begin{table*}[p]
\centering
\begin{threeparttable}
\caption{Piedras preciosas (Valor en mo)}
\begin{tabular}{lcccccccc}
\toprule
$ 1d8$       & 1  & 2  & 3  & 4  &  5  & 6   &  7   &  8 \\
\midrule
Valor(mo) & 10 & 25 & 50 & 75 & 100 & 250 & 500  & 750 + $$ 2d20$x10$\\
\bottomrule
\end{tabular}
\end{threeparttable}
\end{table*}

\begin{table*}[p]
\centering
\begin{threeparttable}
\caption{Clases de Objetos mágicos ($1d8$)}
\begin{tabular}{cccccccc}
\toprule
1  &  2 & 3 & 4 & 5 & 6 &7 & 8 \\    \midrule
Pociones& Anillos & Pergaminos & Varitas & Única & Armas & Armas & Armaduras\\
\bottomrule
\end{tabular}
\end{threeparttable}
\end{table*}

\begin{table*}[p]
\centering
\caption{Objetos Mágicos}
\begin{tabular}{lllll}
\toprule
1d8 & Pociones        & Anillos           & Pergaminos              & Varitas \\
\midrule
1 & Dormir            & INT + 1           & Encantar                & Proyectil mágico \\
2 & Invisibilidad     & FUE + 1           & Oscuridad               & Dormir \\
3 & Levitar           & Invisibilidad     & Toc, toc                & Toc, toc \\
4 & Curar Heridas     & Levitar           & Quitar magia            & Ilusión \\
5 & Agua bendita      & CAR + 2           & Bendecir                & Curar heridas \\
6 & Veneno            & Detectar el mal   & Quitar maldición        & Fétido \\
7 & Curar Enfermedad  & Encontar trampas  & Hablar con los muertos  & Quitar magia \\
8 & Esclavizar        & DES + 1           & Purificar               & Encantar \\
\bottomrule
\end{tabular}
\end{table*}

\vspace{10pt}
\includegraphics[width=\columnwidth]{kidsplay.png}


\section{Diez reglas para jugar mejor}
\begin{enumerate}
\item Conoce las reglas.
\item Acude a la fuerza cuando no se te ocurra nada mejor.
\item Piensa como tu personaje, haz que parezca real.
\item Confía en el máster.
\item No confíes nunca en un goblin sonriente.
\item Ayuda a tus compañeros.
\item Usa la magia de forma creativa.
\item Sé un héroe, valiente y honorable.
\item Pero recuerda que hasta los héroes tienen que huir, a veces.
\item Diviértete
\end{enumerate}

\end{multicols}

\part{Herramientas para el juego}

Lo que sigue no son reglas sino herramientas que \emph{puedes} usar cuando juegues.



\section{El clima}

\begin{table*}[p]
\centering
\caption{Temperatura}
\begin{tabular}{lllll}
\toprule
3d4 & Primavera       & Verano           & Otoño              & Invierno \\
\midrule
2   &
3
4
5
6
7   & Fresco
8   & Fresco
9   & Cálido
10  &
11
12
\bottomrule
\end{tabular}
\end{table*}

\vspace{10pt}
\includegraphics[width=\columnwidth]{kidsplay.png}

\end{document}
