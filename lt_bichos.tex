\documentclass[10pt,spanish,DIV=10,twoside=false]{scrartcl}

% ------ tipografía ---------------------
\usepackage[spanish,es-noindentfirst]{babel}
\usepackage[utf8]{inputenc} % --- codificación de caracteres (entrada)
\usepackage[T1]{fontenc}
\usepackage{Alegreya} 			% --- U otra fuente que se desee
\renewcommand*\oldstylenums[1]{{\AlegreyaOsF #1}}

\usepackage[T1]{fontenc}


\addtokomafont{disposition}{\rmfamily} 	% --- Encabezados en Serif
\addtokomafont{descriptionlabel}{\rmfamily}

\usepackage{microtype}  % --- Arreglos Tipográficos
\usepackage{setspace} % --- Doble espacio, usado en haikus
\usepackage{threeparttable}
\setlength{\columnsep}{1cm}
\usepackage[autostyle]{csquotes}
\usepackage{hyperref}
\usepackage{booktabs}
\usepackage{enumitem}
\usepackage{longtable}

%\pagestyle{plain}
\usepackage{graphicx}
\usepackage{fixltx2e}
\usepackage{multicol}


\makeindex


\begin{document}

\title{¡Monstruos!}
\subtitle{Suplemento de Laberintos Tenebrosos}
\author{Miguel de Luis}
\date{\today}
\maketitle

\section{Monstruos y Criaturas}

Nos gustan los monstruos. Nos gustan tantos que llamamos \emph{monstruo}, sin
pensar, a cualquier criatura del juego. Si me pusiera pijo diría que son
\emph{personajes no jugadores}, que es lo que son: personajes que no llevan los
jugadores, pero es un término que suena administrativo y aburrido. Así que los
llamamos monstruos.

\begin{table*}[p]
\caption{Animales}
\label{c:animales}
\centering
\begin{threeparttable}
\begin{tabular}{lcccp{8cm}}
\toprule
Criatura             & Nivel &   CA & Daño & Notas\\
\midrule
Caballo \\
Jabalí \\
Lobo \\
Gran Felino \\
\midrule

\bottomrule
\end{tabular}

\end{threeparttable}
\end{table*}



\begin{table*}[p]
\caption{Bichos}
\label{c:bichos}
\centering
\begin{threeparttable}
\begin{tabular}{lcccp{8cm}}
\toprule
Criatura             & Nivel &   CA & Daño & Notas\\
\midrule
Escarabajo de Fuego  &  1 & 6 & Deslumbra &  Impacto automático. Tirada de CON o incapaz de ver $ 1d6$ minutos\\
Nube de Diezmil moscas& 2 & 9 & Zumbido & Tirada de SAB o penalización de 1 a todas las acciones $1d6$ minutos\\
Murciélago Vampiro   & 1 & 6 & $ 1d4$ & Salva CON el turno siguiente o Pierdes $ 1d6$ de vida extra.\\
Ciempiés gigante     & 1 & 7 & 0 & Salva CON o quedas insconsciente hasta terminar el combate.\\
\midrule
Hormiga gigante      & 2 & 6 & $ 1d6$ & Veneno. Salva CON o sufre $ 2d6$ más de daño.\\
Mosca gigante        & 2 & 3 & Especial & Apresa a la víctima entre sus patas y la deja caer a 3d4 metros de altura.\\
Horda de Bichos      &  1 &  8  & 1d4 & Impacto automático. Inmunes a las armas (¡Son demasiados!)\\
\midrule
Escorpión Gigante    & 3 & 2 & Especial & $1d6$ (1-4) Ataque de pinza, $1d4$ con Ataque 17 ó (5-6) Ataque de veneno con Ataque 19\\
Araña Gigante        & 3 & 4 & Especial & Ver entrada\\
\midrule
Cubo Gelatinoso      & 4 & 2 &$  d10$ & Salva CON o paralizado. Inmune al frío y rayos.\\

\midrule

\bottomrule
\end{tabular}

\end{threeparttable}
\end{table*}

\end{document}
